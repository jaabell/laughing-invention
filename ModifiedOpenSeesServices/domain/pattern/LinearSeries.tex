%File: ~/OOP/domain/pattern/LinearSeries.tex
%What: "@(#) LinearSeries.tex, revA"

\noindent {\bf Files}   \\
\indent \#include $<\tilde{ }$domain/pattern/LinearSeries.h$>$  \\

\noindent {\bf Class Declaration}  \\
\indent class LinearSeries: public DomainComponent  \\

\noindent {\bf Class Hierarchy} \\
\indent MovableObject \\
\indent\indent TimeSeries \\
\indent\indent\indent {\bf LinearSeries} \\

\noindent {\bf Description} \\ 
\indent The LinearSeries class is a concrete subclass of TimeSeries.
The relationship between the pseudo time and the load factor is linear
for objects of this class. \\


\noindent {\bf Class Interface} \\
\indent // Constructor \\ 
\indent {\em LinearSeries(double factor = 1.0);}\\ \\
\indent // Destructor \\ 
\indent {\em virtual $\tilde{ }$LinearSeries();}\\  \\
\indent // Public Methods \\ 
\indent {\em  virtual double getFactor(double pseudoTime);}\\
\indent {\em  virtual int sendSelf(int commitTag, Channel \&theChannel);}\\
\indent {\em  virtual int recvSelf(int commitTag, Channel \&theChannel,
FEM\_ObjectBroker \&theBroker);}\\
\indent {\em  virtual void Print(OPS_Stream \&s, int flag =0);}\\

\noindent {\bf Constructor} \\ 
\indent {\em LinearSeries(double cFactor = 1.0);}\\ 
The tag TSERIES\_TAG\_LinearSeries is passed to the TimeSeries
constructor. Sets the constant factor used in the relation to {\em
cFactor}. \\

\noindent {\bf Destructor} \\
\indent {\em virtual $\tilde{ }$LinearSeries();}\\ 
Does nothing. \\

\noindent {\bf Public Methods} \\
\indent {\em  virtual double getFactor(double pseudoTime);}\\
Returns the product of {\em cFactor} and {\em pseudoTime}. \\

\indent {\em  virtual int sendSelf(int commitTag, Channel
\&theChannel);}\\
Creates a vector of size 1 into which it places {\em cFactor} and
invokes {\em sendVector()} on the Channel object. \\

\indent {\em  virtual int recvSelf(int commitTag, Channel \&theChannel,
FEM\_ObjectBroker \&theBroker);}\\
Does the mirror image of {\em sendSelf()}. \\

\indent {\em  virtual void Print(OPS_Stream \&s, int flag =0) =0;}\\
Prints the string 'LinearSeries' and the factor{\em cFactor}.